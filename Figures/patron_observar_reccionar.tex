% Patron Obervar y reaccionar
	

\shorthandoff{>}
\begin{tikzpicture}[
node distance=0.5cm,
every node/.style={
	draw,
	rounded corners,
	minimum height=1cm,
	minimum width=2.5cm,
	align=center 
},
label/.style={    % Nuevo estilo para etiquetas
	draw=none,    % Sin borde
	minimum height=0cm,  % Sin altura mínima
	minimum width=0cm   % Sin ancho mínimo
}
]
% Definición de nodos
\node (observer) {proceso\\ observador};
\node[above=0.5cm of observer] (sensors) {sensores};
\node[right=2cm of observer] (analysis) {proceso de\\ análisis};
\node[right=2cm of analysis] (display) {proceso de\\ despliegue};
\node[above=0.5cm of display] (screen) {pantalla};
\node[below left=0.5cm and 0.5cm of analysis] (alarm) {proceso de\\ alarma};
\node[below right=0.5cm and 1cm of analysis] (reactor) {proceso reactor};
\node[below=of alarm] (alarmDevice) {alarma};
\node[below=of reactor] (otherDevice) {otro equipo};

% Conexiones
\draw[-latex] (sensors) -- (observer);
\draw[-latex] (observer) -- node[label, above] {valores de\\ sensor} (analysis);
\draw[-latex] (analysis) -- node[label, above] {valores a\\ desplegar} (display);
\draw[-latex] (display) -- (screen);
\draw[-latex] (analysis) -- (alarm);
\draw[-latex] (alarm) -- (alarmDevice);
\draw[-latex] (analysis) -- (reactor);
\draw[-latex] (reactor) -- (otherDevice);
\end{tikzpicture}
\shorthandon{>}